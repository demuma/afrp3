\documentclass[ti,nofooter]{own-haw-page}
\usepackage[german]{babel}
\usepackage{color}
\usepackage{enumitem}
\usepackage{nameref}
\usepackage{url}
\usepackage{seqsplit}
\usepackage[utf8]{inputenc}
\usepackage[hidelinks]{hyperref}
\usepackage[many]{tcolorbox}
\usepackage{tabto}

\newcommand{\raus}[1]{}
\newcommand{\aufg}{\marginpar{{\em Protokoll!}}}
\newcommand{\back}{\ensuremath{\backslash}}
\newcommand{\pfeil}{\ensuremath{\rightarrow}}
\newcommand{\platzlinie}[1]{%
~\\[#1]
~\\
%$\underline{\hfill}$~\\
\hrule~\\
}

\definecolor{lightblue}{rgb}{0.8, 0.9 ,1}

\newcommand{\arrow}{{\textgreater{}\textgreater} }

% Tilde%%%%%
\newcommand{\addtildealt}{\textasciitilde{}}
% schicker, funktioniert noch nicht:
%\newcommand{\addtilde}{{\Large\textasciitilde{}}}

\newcommand{\addtildealts}{{\fontfamily{ptm}\selectfont\textasciitilde}{}}

\newcommand{\addtilde}{\raisebox{-0.3em}\textasciitilde{}}

% Abstand
\newcommand{\addspace}[1]{\vspace{#1}}

% Codezeile mit $ beginnend
\newcommand{\command}[1]{%
  \vspace{10pt}
  {\ttfamily \fcolorbox{black}{lightblue}{
      \parbox{\textwidth} {
        \vspace{-10pt}
        \def\labelitemi{\$}
        \begin{itemize}[leftmargin=3.5ex, itemsep=-1ex, rightmargin=4ex]
          #1
          \vspace{-10pt}
      \end{itemize}}
  }}\vspace{10pt}}

%%%%%%%%%%%%%%%%%%%%%%%%%%%%%%%%%%%%%%%%%%
% Neues Design (tcolorbox)
\tcbset{settings/.style={
    fontupper=\ttfamily,
    breakable,
    enhanced, 
    colback=lightblue, 
    arc=0mm, 
    boxrule=0.5pt,
    enlarge bottom by=0.5ex,
    enlarge top by=1ex,  
    left=1.5ex,
    right=0ex,
    top=0.4ex,
    bottom=0.4ex}}

% Nichtnummerierter Code (neu)
\newcommand{\codespace}[1]{
  \begin{tcolorbox}[settings]{}
    #1
  \end{tcolorbox}
  \vspace{10pt}
}

% Codezeile mit $ beginnend (neu, unbenutzt)
\newcommand{\codedollar}[1]{
  \begin{tcolorbox}[settings]{}
    \def\labelitemi{\$}
    \begin{itemize}[leftmargin=3ex, itemsep=-1ex]
      #1
    \end{itemize}
  \end{tcolorbox}
}

% Unbenutzt
\newcommand{\codenumber}[1]{
  \begin{tcolorbox}[settings]{}
    \begin{enumerate}[leftmargin=3ex, label={\arabic*}, itemsep=-1ex]
      #1
    \end{enumerate}
  \end{tcolorbox}
}
%%%%%%%%%%%%%%%%%%%%%%%%%%%%%%%%%%%%%%%%%%

% Nichtnummerierter Code (alt)
\newcommand{\codebox}[1]{%
  \vspace{10pt}
  {\ttfamily \fcolorbox{black}{lightblue}{
      \parbox{\textwidth} {
        \vspace{-10pt}
        \def\labelitemi{}
        \begin{itemize}[leftmargin=1ex, itemsep=-1ex, rightmargin=4ex]
          #1
          \vspace{-10pt}
      \end{itemize}}
  }}\vspace{10pt}}

% Normaler nummerierter Code
\newcommand{\code}[1]{
  \vspace{10pt}
  %{\setlength{\fboxrule}{1pt} % Rahmendicke definieren
  {\ttfamily \fcolorbox{black}{lightblue}{
      \parbox{\textwidth} {
        \vspace{-10pt}
        \begin{enumerate}[leftmargin=3.5ex, label={\arabic*}, itemsep=-1ex, rightmargin=4ex]
          #1
          \vspace{-10pt}
      \end{enumerate}}
  }}\vspace{10pt}}
%}

\sloppy 


\begin{document}
%\date{16.~April~2020}
%\date{07.~April~2020}
%\date{07.~Mai~2020}
%\date{29.~Mai~2020}
%\date{16.~January~2021}
\date{08.~November~2022}


%\setcounter{section}{0} % AFRP 1
%\setcounter{section}{1} % AFRP 2
\setcounter{section}{2} % AFRP 3
\newcounter{dummymcpdsv} 
\setcounter{dummymcpdsv}{\value{section}}
\addtocounter{dummymcpdsv}{1}


\begin{page}

\begin{Large}
{\centering \bfseries \"{U}bungsblatt   \arabic{dummymcpdsv}\\ zur Veranstaltung  
\glqq AFR -- Autonomes Fahren und Robotik''\\}
\end{Large}
~\\[-0.5cm]

\section{Navigation und Pfadplanung}

In der \"{U}bung \arabic{dummymcpdsv} geht es um Navigationsstrategien und Pfadplanung innerhalb bekannter und bereits kartierter Umgebungen. \\


\textbf{Zur Bearbeitung der \"{U}bungsaufgaben:} Bearbeiten Sie auf jeden Fall
\textit{alle} \"{U}bungsaufgaben. Ausgenommen hiervon sind lediglich die mit \glqq
optional'' gekennzeichneten Textstellen. Lesen Sie sich die Aufgaben gut durch.
Sollten Sie eine Aufgabe nicht l\"{o}sen k\"{o}nnen, so beschreiben Sie
zumindest, wie weit Sie gekommen sind und auf welche Weise Sie vorgegangen sind.
%
Aufgaben mit der Randbemerkung \textit{``Protokoll!}'' sind 
%Die Aufgaben sind direkt 
im Protokoll zu beantworten.
Als L\"{o}sung der Aufgaben ist %entweder
%alternativ
(je nach Aufgabe)
Programmcode abzugeben
(kommentiert) oder ein kurzer Text.


\section{Programmierung}

In der \"{U}bung 3 geht es um Navigationsstrategien und Pfadplanung innerhalb bekannter und bereits kartierter Umgebungen.

\subsection{Vorbereitungsaufgaben}

\begin{enumerate}

\item Schauen Sie sich einmal die Beschreibungen zu den \textbf{ROS}-packages
\textit{AMCL}, \textit{teb\_local\_planner}, \textit{dwa\_local\_planner} und \textit{global\_planner} im \textit{ROS-Wiki} an. Was wird zu den Algorithmen und zur Funktionsweise gesagt? 

\end{enumerate}

\subsection{Praktikumsaufgaben}


\begin{enumerate}

\item 
Bearbeiten Sie die Schritte \textbf{\nameref{sec:locnav}} (Anhang \ref{sec:locnav}). Wie findest sich das Fahrzeug in der Umgebung zurecht? Beschreiben Sie, was nach der Positionssch\"{a}tzung passiert.
\aufg

\item 
\"{A}ndern Sie den \textit{Global planner} von \textbf{D*} zu \textbf{A*} mittels \textit{use\_quadratic}. Wie beeinflusst diese \"{A}nderung den Pfadplanungsprozess? Probieren Sie die anderen beiden \textit{Local planner}. Wie unterscheiden diese sich voneinander? Optimieren Sie die Parameter. 
\aufg

\item 
Welche Aufgabe haben die lokale und globale Kostenkarte? Ver\"{a}ndern Sie diese mittels \textit{inflation\_radius} und \textit{cost\_scaling\_factor}. Wie wirkt sich diese Parameter aus? 
\aufg

\item 
Bearbeiten Sie die Schritte \textbf{\nameref{sec:carlanav}} (Anhang \ref{sec:carlanav}). Beschreiben Sie den Ablauf der Pfadplanung und Pfadverfolgung im Carla Simulator. Welche Komponenten sind in \textbf{RVIZ} zu sehen? \aufg\

\item 
Betrachten Sie die entsprechende \textit{launch file} unter \textit{carla\_ad\_demo/launch} zur AD Demo. Welche Prozesse werden aufgerufen? \"{U}berlegen Sie, welche Aufgaben diese erf\"{u}llen. 
\aufg

\item 
\"{A}ndern Sie den \textit{spawn\_point} ab und durchlaufen Sie erneut die Demo. Die Werte entnehmen Sie dem Pygame-Fenster an einer Position ihrer Wahl.

\item 
Schauen Sie sich das dazugeh\"{o}rige Python-Skript zur Pfadplanung unter \textit{carla\_waypoint\_publisher/src/carla\_waypoint\_publisher} an. Wie wird hier die Pfadplanung realisiert? Welche Informationen erh"{a}lt das AD von der Simulation? 
\aufg

\item 
Wie unterscheiden sich die Aufgaben des \textit{Global planner} und \textit{Local planner}? Wie ist der \textit{Local planner} unter \textit{carla\_ad\_agent/src/carla\_ad\_agent} im Carla Simulator implementiert? Wie wird hier die Pfadverfolgung geregelt? 
\aufg

\item 
Bearbeiten Sie die Schritte \textbf{\nameref{sec:planningtracking}} (Anhang \ref{sec:planningtracking}). Beschreiben Sie das Verfahren \textit{Pure Pursuit}. Schauen Sie dazu auch in den dazugeh\"{o}rigen Code. 
\aufg

\item 
Optional: Setzen Sie sich mit dem Code Beispiel auseinander. Wie ist hier die die \textit{Pfadplanung} und \textit{-verfolgung} implementiert. 

\end{enumerate}

\appendix


%%%%%%%%%%%%%%%%%%%%%%%%%%%%%%%%%%%%%%%%%
\section{Localization and navigation within known environments}
\label{sec:locnav}

In the previous session we have learned how to generate an occupancy grid map of an unknown environment in the \textbf{Carla}. Now we want to localize ourself within that map and navigate. \\

In this session we will make use of \textit{Adaptive Monte Carla Localization} and the build-in features of the \textbf{Carla}. All required files for this workshop can be found in the ROS package \textit{afrp3}. Just download them from \textbf{Teams} and move it into  the \textit{source} directory of your \textit{catkin workspace}. \\

Append the \textit{.bashrc} from the \textbf{ROS}-package \textit{afrp3} to your \textit{.bashrc} and source it.

\code{
\item echo {\dq source /opt/ros/noetic/setup.bash\dq} \arrow \addtilde/.bashrc
\item echo {\dq source \addtilde/carla\_ws/src/afrp3/.bashrc\dq} \arrow \addtilde/.bashrc
\item source \addtilde/.bashrc
}

\vspace{-6pt}


%%%%%%%%%%%%%%%%%%%%%%%%%%%%%%%%%%%%%%%%%
\subsection*{Localization in grid-based maps with AMCL}


Let's start \textbf{Carla}, the \textbf{Carla ROS Bridge} and \textbf{RVIZ}.

\code{
\setcounter{enumi}{3}
\item /opt/carla-simulator/CarlaUE4.sh \&
\item roscore \&
\item source \addtilde/carla\_ws/devel/setup.bash 
\item roslaunch afrp3 carla\_ros\_bridge.launch
\item roslaunch afrp3 carla\_rviz.launch
}

Our vehicle named \textit{ego\_vehicle} has spawned. Now we need to convert the generated \textit{3D point cloud} to a \textit{2D laser scan} and all \textit{Twist} commands to \textit{Carla Control} commands: 

\code{
\setcounter{enumi}{8}
\item roslaunch afrp3 carla\_laser\_scan.launch
\item roslaunch carla\_twist\_to\_control carla\_twist\_to\_control.launch
}

Load your previously generated map or choose one of the two provided: 

\raus{
\code{
\setcounter{enumi}{10}
\item rosrun map\_server map\_server \addtilde/carla\_ws/src/afrp3/map/carla\_map\_\\\textit{map\_type}.yaml
}
}

\code{
\setcounter{enumi}{10}
\item roslaunch afrp3 carla\_map.launch map\_type:=\textit{hector or gmapping}
}


Start \textit{AMCL}: 

\code{
\setcounter{enumi}{11}
\item roslaunch afrp3 carla\_amcl.launch
}

Since \textit{AMCL} doesn't perform a initial position estimation and just spawns the vehicle at the position predefined in the \textit{launch} file, \textbf{2D Pose Estimate} can set a starting point. It can also be used to reposition´ the vehicle at any given time. \\

The \textbf{RQT Gui} can be used to control the vehicle using \textit{Twist} commands.

\code{
\setcounter{enumi}{12}
\item rosrun rqt\_robot\_steering rqt\_robot\_steering
}

Now, in the upper left corner, change the subscribed topic to \textit{/carla/ego\_vehicle/twist}. And set the maximum/minimum velocity to \textit{10.0/-10.0}. Steering should now work. 

\vspace{-6pt}


%%%%%%%%%%%%%%%%%%%%%%%%%%%%%%%%%%%%%%%%%
\subsection*{Path planning and tracking with move\_base}

To autonomously navigate within the known environment, we need a \textit{global planner} and \textit{local planner} with their respective costmaps. The global path planning the algorithms \textit{A*} and \textit{D*} are available. For local path planning the packages \textit{base\_planner}, \textit{DWA\_local\_planner} and \textit{TEB\_local\_planner} can be used. The costmaps will help to find the optimal path to the target position while maintaining the lowest costs. \\


By default, \textit{A*} is set for global planning. Set \textit{DWA} for local planning. Now, let's start the navigation:

\code{
\setcounter{enumi}{13}
\item roslaunch afrp3 carla\_move\_base.launch dwa:=true
}

Use \textbf{2D Nav Goal} to set a target position.

\vspace{-6pt}


%%%%%%%%%%%%%%%%%%%%%%%%%%%%%%%%%%%%%%%%%
\subsubsection*{Editing parameters}

All parameters within \textbf{ROS} can be seen and modified in real-time via \textbf{RQT}:

\code{
\setcounter{enumi}{14}
\item rosrun rqt\_gui rqt\_gui
}

\textbf{Note:} This will only temporarily change the settings! But the effects of the change will show immediately! \\

\vspace{-6pt}

Parameter settings for the \textit{global} and \textit{local costmaps} can be found in:

\code{
\setcounter{enumi}{15}
\item cd \addtilde/carla\_ws/src/afrp3/param
\item gedit global\_costmap\_params.yaml
\item gedit local\_costmap\_params.yaml
}

The \textit{inflation\_radius} and \textit{cost\_scaling\_factor} will have a significant influence on the costmaps and thus on the path planning process. \\

To choose between the \textit{A*} and \textit{D*} algorithm for the global navigation, edit the following param file:

\code{
\setcounter{enumi}{18}
\item gedit global\_planner\_params.yaml
}

The local planner can be changes by setting the arguments \textit{dwa}, \textit{teb} or \textit{base} to \textit{true}. The respective parameter file will be loaded as well:

\code{
\setcounter{enumi}{19}
\item roslaunch afrp3 carla\_move\_base.launch dwa:=true \textit{or} teb:=true \textit{or} base:=true
}

Start \textbf{RQT} to change parameters:

\code{
\setcounter{enumi}{20}
\item rosrun rqt\_reconfigure rqt\_reconfigure -""-force-discover
}

and confirm your changes with \textbf{ENTER}. Hit \textbf{Refresh} in case parameters are missing. \\

\vspace{-6pt}

\textbf{Note:} Changes parameters will not be saved!\\

\vspace{-6pt}

The commands for localization with \textbf{AMCL} and navigation with \textbf{move\_base} can be started simultaneously using the provided launch file \textit{start\_carla\_navigation.launch}. \\

\vspace{-6pt}

Just type:

\code{
\setcounter{enumi}{21}
\item roslaunch afrp3 start\_carla\_navigation.launch slam\_method:=\textit{hector or gmapping} dwa:=true
}


%%%%%%%%%%%%%%%%%%%%%%%%%%%%%%%%%%%%%%%%%
\section{Using the build-in navigation in Carla Simulator}
\label{sec:carlanav}

\subsection*{The Autonomous Driving Demo}

Start the \textbf{Carla Simulator} with your graphic settings, the \textbf{roscore} and \textbf{RVIZ}:

\code{
\setcounter{enumi}{22}
\item /opt/carla-simulator/CarlaUE4.sh\&
\item roscore \&
\item rosrun rviz rviz -d \addtilde/carla\_ws/src/afrp3/rviz/carla\_ad\_demo.rviz \&
\vspace{-12pt}
}

Start the demo:

\code{
\setcounter{enumi}{25}
\item roslaunch carla\_ad\_demo carla\_ad\_demo.launch
}

Now, the generated path as well as the current and target pose of the vehicle can be seen. Throughout the path tracking process, both are updated. \\

The respective \textit{launch file} can be viewed and edited in:

\code{
\setcounter{enumi}{26}
\item gedit \addtilde/carla\_ws/src/ros-bridge/carla\_ad\_demo/launch/carla\_ad\_demo\\.launch
}

To take a deeper look into how the path planning is implemented in this demo, the respective python script has to be viewed in:

\code{
\setcounter{enumi}{27}
\item gedit \addtilde/carla\_ws/src/ros-bridge/carla\_waypoint\_publisher/src/carla\\\_waypoint\_publisher/carla\_waypoint\_publisher.py
}


%%%%%%%%%%%%%%%%%%%%%%%%%%%%%%%%%%%%%%%%%
\section{Path planning and tracking}
\label{sec:planningtracking}

As you have seen in the previous chapters, the \textbf{Carla} offers the possibility to interact with its simulation environment via a \textit{PythonAPI}. Examples are at:

\code{
\setcounter{enumi}{28}
\item cd /opt/carla-simulator/PythonAPI/examples
}

A full documentation of all features can be found at: \\

\url{https://carla.readthedocs.io/en/latest/python_api/} \\

For more tutorials, visit: \\

\url{https://carla.readthedocs.io/en/0.9.7/python_api_tutorial/} \\

\textbf{Note:} Some tutorials may be outdated!


%%%%%%%%%%%%%%%%%%%%%%%%%%%%%%%%%%%%%%%%%
\subsection*{Lateral and longitudinal control for path tracking}

The following algorithms for \textit{lateral control} or ``steering'' of the vehicle in \textit{path tracking} are commonly used in Autonomous Driving:

\begin{enumerate}

\item Pure Pursuit
\item Model predictive control (MPC)
\item Linear-quadratic regulator (LQR)
\item Stanley Controller (Stanford University)

\end{enumerate}

A PID-controller is used for \textit{longitudinal control}. \\

This workshop contains a small demonstration that shows how the \textit{Pure Pursuit}-Algorithm works. \\

To start the very simplified simulation, go to the \textit{Pure Pursuit example} in \textit{examples} folder and start the simulation:

\code{
\setcounter{enumi}{29}
\item cd \addtilde/catkin\_ws/src/afrp3/examples/pure\_pursuit
\item python sim.py
}

Navigation specific python scripts, that are implemented in \textit{Carla Simulator} can be found in:
 
\code{
\setcounter{enumi}{31}
\item cd /opt/carla-simulator/PythonAPI/carla/agents/navigation
}

The lateral and longitudinal controller is implemented in \textit{controller.py}. A closer look into these files will help to understand how the general navigation is implemented. 


%%%%%%%%%%%%%%%%%%%%%%%%%%%%%%%%%%%%%%%%%
\subsection*{Code Example}

This seesion also includes a basic code example for navigation using the \textbf{PythonAPI}. That includes python scripts for setup of the simulation environment, the vehicle and the camera. \\

Start the simulation:

\code{
\setcounter{enumi}{32}
\item cd \addtilde/carla\_ws/src/afrp3/examples/carla\_pure\_pursuit
\item python citymap.py
\item python camera.py
\item python simpleclient.py
}

The original version can be found at: \url{https://github.com/copotron/sdv-course}

A working example can be found:

\code{
\setcounter{enumi}{36}
\item cd \addtilde/carla\_ws/src/afrp3/examples/carla\_navigation
\item python carla\_example.py
}


%%%%%%%%%%%%%%%%%%%%%%%%%%%%%%%%%%%%%%%%%
\section{Higher functions of Autonomous Driving in Carla Simulator}

Carla has developed a \textbf{scenario runner} to simulate predefined traffic scenarios. \\

Just download the latest \textit{Scenario Runner} from \textbf{github} to your \textit{catkin source} folder and build the entire folder again.

\code{
\setcounter{enumi}{38}
\item git clone https://github.com/carla-simulator/scenario\_runner
\item cd \addtilde/carla\_ws
\item rm -rf build/ devel/
\item catkin\_make
}

To start, go to the scenario directory, set the scenario and start manual control:

\code{
\setcounter{enumi}{42}
\item cd src/scenario\_runner
\item ./scenario\_runner.py -{-}scenario FollowLeadingVehicle\_1 -{-}reloadWorld
\item ./manual\_control.py
}

A list of scenarios can be found here: \\
\url{https://carla-scenariorunner.readthedocs.io/en/latest/list_of_scenarios/} \\

\vspace{-6pt}

Full documentation can be found here: \\ \url{https://carla-scenariorunner.readthedocs.io/en/latest/}

\end{page}
\end{document}